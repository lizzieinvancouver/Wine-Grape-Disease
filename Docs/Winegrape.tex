\documentclass[11pt,letter]{article}
\usepackage[top=1.00in, bottom=1.0in, left=1.1in, right=1.1in]{geometry}
\renewcommand{\baselinestretch}{1.1}
\usepackage{graphicx}
\usepackage{natbib}
\usepackage{amsmath}

\def\labelitemi{--}
\parindent=0pt

\begin{document}


\title{Winegrape Disease}
\author{Darwin Sodhi, Jonathan Davies & Elizabeth Wolkovich }
\date{October 2019}


\maketitle

\section{Introduction}
Pathogens play an important role in structuring communities (Janzen 1970, Connell 1978, Lafferty \textit{et al}. 1997) By definition, pathogens reduce the fitness of their hosts, and this in turn can influence how species coexists in a community (Mordecai. 2011). For example, this reduction in species’ fitness is thought to stabilize communities by reducing dominance of one species (Chesson. 2000). Pathogens are also though to promote diversity via the elevating seedling mortality closer to conspecific adult trees due to higher pathogen pressure (HilleRisLambers \textit{et al}. 2002) – the Janzen-Connell hypothesis. The Janzen Connell hypothesis has been widely studied in the tropics (Clark. 2010), but its importance in structuring temperate plant communities has been less –well studied. 
\paragraph{}Plant pathogens can cause mortality, rapid declines of populations or large shifts in the structure of plant communities (Gilbert. 2002). In agricultural settings, where there can be dense stands of conspecifics, pathogen outbreaks can be particularly devastating. A better understanding of the ability for a pathogen to infect new hosts would allow for predictions on future disease pressure on communities (Borer \textit{et al}. 2016). Such predictions on future disease pressure can help in mitigating the potential of negative impacts of novel plant-pathogen interactions (Parker and Gilbert. 2004). For example, the bacterium \textit{Xylella fastidiosa}, primarily regarded as a disease of olive trees (\textit{Olea europaea}), was first described on trees in Italy, but was subsequently reported in parts of Europe in 2013 (Almedia. 2016). Because \textit{Xylella fastidiosa} was known to be able to between multiple hosts, the European Commission immediately implemented an eradication program; nonetheless, the disease still managed to significantly impact the agricultural output of several European countries (Almedia. 2018).Knowing the host range of \textit{Xylella fastidiosa} helped mitigate impacts by ensuring a timely and widespread response, without such knowledge there would have been a myriad of potential negative impacts (i.e: ecological, economic and social) for many of the infected areas.  The case study of \textit{Xylella fastidiosa} shows why it is imperative to understand what determines pathogen host breadth.

\paragraph{}Parker and Gilbert (2004) suggest that four factors are particularly critical in determining the probability of a host shift: (a) the degree of dependence of the pathogen on live hosts, (b) the degree of specialization of the pathogen, (c) the phylogenetic distance between the novel potential host and hosts with which the pathogen is familiar, and (d) the degree of ecological association between the pathogen and the potential host. The degree of dependence of the pathogen on live hosts can be related to the dispersal mode of the pathogen (Roy. 2001). For example, many papers have shown that pathogens can still spread despite being grown on dead tissue, however this is still overlooked in many ecological studies on pathogen spread (Parker & Gilbert. 2004). The specificity of a pathogen is a product of the range of hosts a pathogen infects (Parker and Gilbert. 2004): highly specialized pathogens are less likely to be found on a wide range of hosts (Alberts 2002). Phylogenetic distance between new host and previous hosts has been shown to be a strong predictor of host shifts (Farr \textit{et al}. 2004, Mack. 1996, Duncan and Williams. 2002). This has been hypothesized to be because phylogenies can capture ecologically relevant information on host physiology, immunology, and life history (Davies & Pederson. 2008, Gilbert & Webb. 2007). However, unpredictable host-pathogen associations are known to occur despite strong phylogenetic patterns underlying most interactions (Weste and Marks. 1987), proximity of hosts and pathogens, abiotic interactions needed for host shifts, and temporal overlap of hosts and pathogens (Parker and Gilbert. 2004), can all mediate the outcome of potential host shifts.

\paragraph{}Phylogeny has been shown to be a good predictor of pathogen host range across various systems such as fungal pathogens of trees (Gilbert & Webb. 2007), fish (Poulin. 1995), and primates (Davies and Pedersen. 2008). However, a diversity of phylogenetic metrics populate the literature, and metric choice requires careful consideration (Tucker \textit{et al}. 2017). Common metrics, such as PD (phylogenetic diversity), and ED (evolutionary distinctiveness) are frequently used in conservation biology (Isaac. 2007), whereas pairwise distance metrics, such as MPD and MNTD, which are used to understand how phylogenetically related species are in a phylogeny, and how this relates to a null expectation (Kembel. 2010) are more common in community ecology (Webb \textit{et al}. 2008) and macroecology (Davies & Buckley. 2011; Tucker \textit{et al}. 2017). More recently, related phylogenetic measures have been used to understand pathogen and host phylodynamics (Fountain-Jones et al. 2017) and pathogen impacts on their hosts (Farrell & Davies 2019). Here, following Gilbert & Webb (2007) and Davies & Pedersen (2008), we use pairwise distance metrics to describe the host breadth of Winegrape pathogens. Limited knowledge on pathogen infections in wild species, especially a lack of data on non-infected hosts (Parker \textit{et al}. 2015), has made it challenging to test predictions. Large databases and macro-ecological approaches have provided one way forward (see e.g. Stephens \textit{et al}. 2016). The more extensive data available for domesticated species provides an alternative approach (see e.g. Farrell & Davies. 2019). Here we examine pathogens of Winegrape (\textit{Vitis vinifera}).

\paragraph{}Agricultural species provide a useful study system because agricultural plant pathogens are often well described. In addition, understanding threats posed by potential host shifts is critically important for food security (Bommarco \textit{et al}. 2013), and breeding for pathogen resistance is a top priority for cultivators. Winegrapes are a particularly useful model system because they are widely planted around the globe (International Organization of Vine and Wine. 2017), they are susceptible to diverse pathogens differing in host breadth (Armijo \textit{et al}. 2016), and climate change is expected to result in susceptibility to a growing number of pathogens in the future (Orduna. 2010). Winegrapes are also an economically important crop (International Organization of Vine and Wine. 2017), and there has been much research on individual Winegrape diseases (Armijo \textit{et al}. 2016, Zahavi \textit{et al}. 2000, Hopkins and Purcell. 2007), providing a wealth of baseline data on important pathogens that affect Winegrape yield and health. 

\paragraph{}In this paper we use three phylogenetic measurements, mean pairwise distance (MPD), mean nearest taxon distance (MPD) and focal distance to agricultural host species (Vitis vinifera), to capture the phylogenetic structure of 47 winegrape pests across over three thousand agricultural host species. We then describe the relationship between XXXXX and XXXX, and explore how pathogen traits (type of pathogen, body size, host specificity and reproductive rate) influence the phylogenetic structure of their distribution across hosts. If we can better predict emergence of winegrape pathogens, we can guide proactive surveillance of high-risk threats, and better prepare to mitigate their negative impacts.


\section{Methods}
\subsection{Data Collection}
We used three online resources (www.webofscience, www.cabi.org and www.scalenet.info) to build a preliminary list of the major pathogens affecting winegrapes (Vitis vinifera) and the list of agricultural host species that they also infect. We queried each database using the following search terms: wine grape pathogens, Vitis vinifera pathogens, and wine grape pathogen impacts, and recorded data for all species for which we could obtain published data on infection rates of V. vinifera and impacts on yield. This returned a list of forty-seven (see supplementary list) pathogens, which we separated into four broad categories: fungal, bacterial, viral and pest. We cleaned taxonomic nomenclature, recorded all synonyms, and obtained a list of all hosts using information from http://www.dpvweb.net/,http://nemaplex.ucdavis.edu/, https://nt.ars-grin.gov/fungaldatabases/). The total host range for each pathogen was obtained by concatenating the list of unique hosts, ensuring that the final host list for each of our forty-seven pathogens included Vitis vinifera. 
For each pathogen, we obtained georeferenced locations and native geographic distribution from cabi.org. Finally, conducted a literature search in Web Of Science (https://www.webofknowledge.com) identify how many papers had been published on each of our final set of 47 pathogens. We recorded the number of the total papers published, the number of papers including all synonyms, and number of papers in agricultural categories.  

\subsection{Host species and phylogeny}
To characterise the phylogenetic distribution of winegrape pathogens on their agricultural hosts, we first, obtained a list of agricultural host species (n = 944) from Milla et al. (2018). We then matched our host list to this subset, returning the list of agricultural hosts infected by each winegrape pathogen. Second, we subset the more inclusive and calibrated phylogenetic tree from Zanne et al. (2014) to just the agricultural host species listed above. Because some hosts were only identified to genus, we performed two analyzes, one that assumed such pathogens infected all agricultural species in that genus, and another assuming only a single arbitrary host species within the genus was infected (ensuring that this species was represented in the Zanne et al. phylogeny). By assuming all host species in a genus are infected, we likely inflate taxonomic clustering of pathogen host range however, using only one single arbitrary specie would likely deflate taxonomic clustering of pathogen host range. By conducting the two analyses congruently we can better infer patterns by comparing and contrasting patterns found between the two analyses.  
\subsection{Phylogenetic metrics}
We used the PICANTE package in R to quantify the phylogenetic clustering of host species infected by each pathogen. Using the subset phylogeny from Zanne et al.(2014), we calculated the mean pairwise distances (MPD) between all hosts for each pathogen, the mean nearest taxon distance (MNTD), which describes the minimum pairwise distance separating hosts for each pathogen, and the mean and minimum distance to Vitis vinifera. Standard effect sizes (SES) were calculated for each measure using the following equation, assuming a null model of no phylogenetic structure (option: tip swap in PICANTE), and 999 replicates. 
Standard Effect Size is Measured:  
\begin{align}
SES_{metric} & = \frac{Metric_{observed}-{mean({Metric_{null})}}}{SD({Metric_{null})}}
\end{align}

We explore variation in phylogenetic structure across pathogens using regression models and Bayesian analyses. First, we tested for broad differences across taxonomic groupings, separating pathogens into four categories: fungal, bacterial, viral and pest. Second, we evaluated differences between specialist (pathogens infecting only a single genus) and generalist (pathogens infecting multiple genera) pathogens. Finally, we tested for trait differences, using body size information for all nematodes and pests using websites: http://nemaplex.ucdavis.edu/ and cabi.org. Equivalent data for body size for bacteria and fungi are not available.All statistical analyses were carried out using the R statistical software (R Development Core Team, 2017).

\end{document}
